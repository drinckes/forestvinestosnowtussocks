\chapter{The Plants of other Southern Lands}

In this chapter, as a necessary background to the consideration of the origins and history of New Zealand plants in the last chapter, we will review fairly briefly the plant coverings of certain southern islands and continents which have strong botanical links with New Zealand.
The Hawaiian Islands, although north of the equator, will also be included as much of their flora appears to have been derived from the south-west Pacific.

\section{Australia}

Most of Australia is much drier than New Zealand and as a consequence most Australian vegetation types and their floras are very different from ours.\footnote{\cite{beadle1981vegetation}}\footnote{\cite{groves1981australian}} Vegetation types with counterparts in New Zealand --- moist closed forests and open alpine communities --- are restricted to the wetter and more mountainous eastern fringe of the continent and Tasmania.
Even here it is estimated that prior to the arrival of man only about 1 per cent of Australia was occupied by closed forests and much less than that by alpine communities.

\subsection{Closed Forests}

These extend in scattered patches from north Queensland to Victoria and Tasmania.
The closed forests without \BotanicRef{Nothofagus} have been divided latitudinally into tropical, subtropical and warm temperate.
The first is found in north Queensland and is characterised by leaves of mesophyll size; the second ranges from central Queensland to central New South Wales and has a predominance of notophylls (small mesophyll); and the last straddles southern New South Wales and eastern Victoria and is predominantly microphyllous.\footnote{\cite{webb1959physiognomic}}
A number of the genera prominent in these forests are shared with New Zealand, for example \BotanicRef{Elaeocarpus}, \BotanicRef{Beilschmiedia}, \BotanicRef{Dysoxylum}, \BotanicRef{Syzygium} and the vines \BotanicRef{Freycinetia}, \BotanicRef{Ripogonum} and \BotanicRef{Parsonsia}, but in most cases there are many more Australian than New Zealand species.
Woody vines and vascular epiphytes are conspicuous in the more northern of these Australian forests, but are less common in the south at the same latitudes as New Zealand's North Island.
In Queensland `strangling' figs may occur as emergents and, in places, species of \BotanicRef{Agathis} and the related \BotanicRef{Araucaria}.
In Queensland also the abundant nest epiphytes are ferns --- species of \BotanicRef{Asplenium}, \BotanicRef{Drynaria} and \BotanicRef{Platycerium} (Stags Horn Fern) --- equivalent ecologically to the asteliad nests of New Zealand.

Closed forests where species of \BotanicRef{Nothofagus} dominate or co-dominate are scattered through south-east Australia near the coast and Tasmania.
In eastern Victoria and Tasmania the species involved, \BotanicRef{Nothofagus cunninghamii}[Nothofagus][cunninghamii], is closely related to \IDX{silver beech}[beech!silver] (\BotanicRef{Nothofagus menziesii}[Nothofagus][menziesii]) of New Zealand. \BotanicRef{Nothofagus moorei}[Nothofagus][moorei] of montane sites in north central New South Wales and near the New South Wales, Queensland border also belongs to the `\BotanicRef{Nothofagus menziesii}[Nothofagus][menziesii] group', but has larger leaves than either \BotanicRef{Nothofagus cunninghamii}[Nothofagus][cunninghamii] or \BotanicRef{Nothofagus menziesii}[Nothofagus][menziesii].

As in New Zealand the \BotanicRef{Nothofagus} forests in Australia have relatively few species and few or no lianes and vascular epiphytes.
Several other tree species are often associated with \BotanicRef{Nothofagus} including \BotanicRef{Ceratopetalum apetalum}[Ceratopetalum][apetalum] (perhaps the counterpart of \BotanicRef{Weinmannia} of the same family in New Zealand) \BotanicRef{Atherosperma moschata}[Atherosperma][moschata] and in Tasmania, the conifers \BotanicRef{Dacrydium franklinii}[Dacrydium][franklinii] and \BotanicRef{Phyllocladus aspleniifolius}[Phyllocladus][aspleniifolius].
Tree ferns are common, particularly in canopy gaps and gulleys.

\subsection{Open Forest}

In this category trees grow sufficiently far apart so their crowns do not form a continuous canopy and the ground as a consequence is not heavily shaded.

In tall open forest, also known as wet sclerophyll, very tall species of \BotanicRef{Eucalyptus} are dominant.
One of these, \BotanicRef{Eucalyptus regnans}[Eucalyptus][regnans], is the tallest flowering plant in the world, attaining a height of \SI{100}{\metre} or more.
Tall open forest is restricted to maritime south-east Australia, Tasmania, and a small area at the tip of south-west Australia.
Where soils are relatively fertile, ferns predominate in the ground cover and where it is infertile, heath-like shrubs, belonging to the Myrtaceae, Proteaceae, Rutaceae and other families predominate.
This forest type is restricted to fairly moist climates and perhaps in the absence of fire it would be succeeded by closed forest.

Low open forest occupies drier sites, sometimes further inland than those of tall open forest, in eastern, northern and south-eastern Australia and Tasmania.
Relatively short species of \BotanicRef{Eucalyptus} predominate in the tree stratum together with species of other genera in the same family and other drought-resistant species such as the scale-leaved Casuarinas.
On fertile sites, tall grasses form the ground cover and on infertile sites heath-like shrubs.
Sometimes trees are absent and pure grassland or heathland results.

\subsection{Tall Shrubland}

In the southern parts of South and Western Australia there is a special type of shrubland known as Mallee.
Here the dominant eucalypts are multistemmed and regenerate after fire from bulky woody tubers known as lignotubers.
Generally inland from the eucalyptus-dominated communities, where climates are drier and soils infertile, drought-resistant species of \BotanicRef{Acacia}, particularly \BotanicRef{Acacia aneura}[Acacia][aneura] (mulga), prevail.
The sometimes sparse ground cover comprises grasses and shrubs.

\subsection{Hummock Grassland}

Succeeding the Acacia shrubland in the arid heart of Australia is a grassland dominated by the prickly small tussocks of grasses, mainly belonging to the genus \BotanicRef{Triodia}.

\subsection{Chenopod Low Shrubland}

In south central Australia on calcareous, sometimes semi-saline, soils there is a low shrubland where the salt-tolerant family Chenopodiaceae is strongly represented.

\subsection[Subalpine and Alpine Communities]{Subalpine and Alpine Communities\thinspace\footnote{\cite{harris1970alpine}}}

Despite its size, the continent of Australia has generally low relief and it is only in the south-east and in Tasmania that mountains rise above treeline.
The mountains are not of alpine form, but are basically raised plateaus with fringing scarps.
Glaciation was extensive during the Ice Age in Tasmania and as a result there are more steep and rocky sites in the mountains there than on the mainland.

The zone \SIrange{400}{500}{\metre} below treeline is occupied by woodland dominated by a few small species of \BotanicRef{Eucalyptus} (Snowgums).
In Tasmania there are, in addition, thickets of a shrubby deciduous species of \BotanicRef{Nothofagus}, \BotanicRef{Nothofagus gunnii}[Nothofagus][gunnii], which belongs to the \IDX{red beech}[beech!red] (\BotanicRef{Nothofagus fusca}[Nothofagus][fusca]) group.
Associated with the \BotanicRef{Nothofagus} are species of \BotanicRef{Athrotaxis}, the only southern hemisphere genus of the conifer family Taxodiaceae.

Frosty valley floors below and above treeline are occupied by a tussock grassland with species of \BotanicRef{Poa}, \BotanicRef{Danthonia} and other genera.

Above treeline, particularly on the mainland, an important plant community is a herbfield of short grasses and such genera as \BotanicRef{Celmisia}, \BotanicRef{Craspedia} and \BotanicRef{Euphrasia} growing on fairly deep, well-drained soils.
Below persistent snow patches is an even shorter herbfield with species of \BotanicRef{Plantago}, \BotanicRef{Neopaxia}, \BotanicRef{Caltha} and \BotanicRef{Ranunculus} as in New Zealand.
Less common on the mainland, but well developed in Tasmania is a shrubbery on rocky sites comprising genera shared with the heaths of the lowlands.

Poorly drained sites support bogs with \BotanicRef{Sphagnum} moss, a variety of small shrubs, sedges and small species of \BotanicRef{Astelia}.
In Tasmania alone cushion bogs with species the same as or related to those of New Zealand are frequent.
Species present include \BotanicRef{Donatia novae-zelandiae}[Donatia][novae-zelandiae] and \BotanicRef{Phyllachne colensoi}[Phyllachne][colensoi] (shared with New Zealand) and species of \BotanicRef{Oreobolus} and \BotanicRef{Astelia}.

Above snow patches comes a type of fellfield with low species of \BotanicRef{Coprosma} and \BotanicRef{Colobanthus} and, on wind-exposed rounded ridge crests, a different type with, among other genera, \BotanicRef{Chionohebe} and \BotanicRef{Drapetes}, which are also shared with New Zealand.
From photographs the latter community looks not unlike the summit plateau fellfield of Macquarie Island.

\section[New Guinea]{New Guinea\thinspace\footnote{\cite{johns1982plant}}\footnote{\cite{wardle1973newguinea}}}

Narrowly separated to the north, the island of New Guinea contrasts with most of Australia in many respects --- it has a moist tropical climate leading to a predominance of closed forests.
Except for extensive plains in the south, it is much more mountainous and the mountains are steep and high enough, even at such low latitudes, to rise above treeline.

\subsection{Lowland Zone}

Along estuaries \IDX{mangrove} forests are well developed and contain many species.
Further inland on the plains, swamp forests are widespread, and in the south, where there is a distinct dry season, monsoon forests characterised by a proportion of deciduous trees occupy better drained sites.
In the driest areas fire is an important factor and an open forest of Australian aspect and relationships covers the landscape.
Species of \BotanicRef{Eucalyptus} and other genera of the Myrtaceae dominate in these forests.
However, the most widespread forest type up to \SI{700}{\metre} is a species-rich, evergreen rain forest.
Strangling figs may occur as emergents over a main canopy of buttressed trees up to \SI{45}{\metre} in height.
Vines and epiphytes are common, the latter including ferns and many orchids.
New Zealand shares only a few genera with these forests, for example, \BotanicRef{Agathis}, \BotanicRef{Elaeocarpus}, \BotanicRef{Dysoxylum} and \BotanicRef{Beilschmiedia}.

\subsection{Montane Zone}

The montane zone ranges from about \SIrange{700}{3200}{\metre} altitude.
The lower montane forests reach up to an average altitude of 1700 m, are \SIrange{20}{25}{\metre} in height and are generally without emergents.
Palms and tree ferns are uncommon.

Some of the tree genera shared with New Zealand are \BotanicRef{Elaeocarpus}, \BotanicRef{Litsea}, \BotanicRef{Weinmannia}, \BotanicRef{Pittosporum} and \BotanicRef{Schefflera}.
Shared vine genera are \BotanicRef{Parsonsia} and \BotanicRef{Clematis}.

In places dense stands of \BotanicRef{Araucaria cunninghamii}[Araucaria][cunninghamii] and \BotanicRef{Araucaria hunsteinii}[Araucaria][hunsteinii] are emergent over lower montane forest species.
The Araucarias establish in forest openings, but eventually give way to broadleaves in the course of forest succession.

`Oak forests', which are also found in the lower montane zone, generally on ridge crests, are dominated by relatively small-leaved, evergreen species of the oak-related genera \BotanicRef{Castanopsis} and \BotanicRef{Lithocarpus}.
Oak forests are usually fairly open with few epiphytes and lianes.
The rather sparse subcanopy includes tree ferns, \BotanicRef{Phyllocladus hypophyllus}[Phyllocladus][hypophyllus] and species of \BotanicRef{Rhododendron} and \BotanicRef{Vaccinium}.

Mid montane forests, from about \SIrange{1700}{2800}{\metre}, are moister than those at lower montane levels and have an abundance of terrestrial and epiphytic ferns, mosses and lichens.
The commonest forest type at this level can be termed conifer broadleaf as conifers are frequent.
They include species of \BotanicRef{Podocarpus}, \BotanicRef{Dacrycarpus}, \BotanicRef{Papuacedrus} (related to \BotanicRef{Libocedrus}) and \BotanicRef{Phyllocladus}.
Among broadleaf genera, species of \BotanicRef{Weinmannia} are particularly frequent.

Species of \BotanicRef{Nothofagus} of the `\BotanicRef{Nothofagus brassii}[Nothofagus][brassii] group' are an important component in these forests, particularly on ridge crests where pure or nearly pure stands of \BotanicRef{Nothofagus} occur.

We also recognise an upper montane conifer broadleaf forest in which liverworts and lichens are abundant on trunks and branches.
Neither \BotanicRef{Weinmannia} nor \BotanicRef{Nothofagus} are present here.

\subsection{Subalpine Zone}

Unlike in New Zealand, \BotanicRef{Nothofagus} does not extend up to the treeline.
The subalpine zone from about \SIrange{3200}{3800}{\metre} is occupied by a low forest about \SI{10}{\metre} high with emergents (mostly conifers) to 15 m.
Again there is a mixture of conifers (\BotanicRef{Dacrycarpus}, \BotanicRef{Papuacedrus} and \BotanicRef{Phyllocladus}) and broadleaves among which species of \BotanicRef{Tittosporum}, \BotanicRef{Myrsine}, \BotanicRef{Schefflera}, \BotanicRef{Rhododendron} and \BotanicRef{Vaccinium} are prominent.

\subsection{Alpine Zone}

Above the subalpine forest and in frosty hollows within it are tussock grasslands similar in appearance to those of the New Zealand mountains.
However, no species of the snowgrass genus (\BotanicRef{Chionochloa}) are present.
Instead the tussocks belong to such grass genera as \BotanicRef{Poa}, \BotanicRef{Deschampsia} and \BotanicRef{Deyeuxia}, also represented in New Zealand and many other parts of the world.
Scattered through the tussock grassland are several shrubs including species of \BotanicRef{Coprosma}, \BotanicRef{Gaultheria} and \BotanicRef{Parahebe} and small trees of \BotanicRef{Olearia spectabilis}[Olearia][spectabilis].
In places clumps of the fern \BotanicRef{Papuapteris linedris}[Papuapteris][linedris] intermingle with the tussocks and at lower elevations a distinctive species of tree fern (\BotanicRef{Cyathea atrox}[Cyathea][atrox]) is scattered throughout.
Above about \SI{4000}{\metre} there is a short grassland with species of \BotanicRef{Festuca} and \BotanicRef{Poa} as well as a number of dwarf shrubs.
In poorly drained places there are swamps and bogs: the swamps dominated by grasses and sedges; the bogs by cushion plants and small shrubs.
The cushion plants include \BotanicRef{Astelia papuana}[Astelia][papuana], \BotanicRef{Oreobolus pumilio}[Oreobolus][pumilio] and \BotanicRef{Carpha alpina}[Carpha][alpina].
The last species also occurs in Australia and New Zealand.

Above about \SI{4200}{\metre} species of \BotanicRef{Styphelia}, \BotanicRef{Parahebe} and \BotanicRef{Drapetes} form close mats on rocky sites with small grasses and species of \BotanicRef{Ranunculus}, \BotanicRef{Parahebe} and \BotanicRef{Potentilla} on level areas and in hollows.

The alpine flora of New Guinea is not rich in species and has links with both north and south temperate regions.
An unusually high proportion, 30 per cent, of the species are found outside New Guinea.

\section{New Caledonia and Other Pacific Islands}

\subsection[New Caledonia]{New Caledonia\thinspace\footnote{\cite{schmid1981fleurs}}}

Although only \SI{400}{\kilo\metre} long, \IDX{New Caledonia} is composed of continental rocks similar to those of New Zealand and has a rich and, in many ways, remarkable flora.
Conifers are particularly prominent with Araucarias, tall and pencil-like or of candelabra form, providing a distinctive character to many landscapes.
Indeed, of the 19 species of \BotanicRef{Araucaria} recognised for the world 13 are restricted to \IDX{New Caledonia}.
Several species of \BotanicRef{Agathis} (\IDX{kauri}) are also present, several genera of the family Podocarpaceae, including the only known parasitic conifer (\BotanicRef{Parasitaxus ustus}[Parasitaxus][ustus]) and 3 species of \BotanicRef{Libocedrus}.
Several old southern families of flowering plants are also strongly represented --- Myrtaceae, Proteaceae, Epacridaceae, Cunoniaceae.
The primitive family Winteraceae is centred in \IDX{New Caledonia} with 4 genera and about 16 species.

For the most part \IDX{New Caledonia} has a quite steep topography, but the mountains up to \SI{1600}{\metre} in altitude are not high enough to support alpine vegetation.
On the drier west of the island and on burnt sites in the east there are extensive open woodlands of Australian aspect dominated by \BotanicRef{Melaleuca quinquenervia}[Melaleuca][quinquenervia] with its spongy fire-resistant bark.
As in Australia, the \BotanicRef{Melaleuca} also grows in swampy sites and it has been suggested that prior to the arrival of human beings it may have been largely restricted to such places, becoming more widespread following destruction of rain forests on better drained sites by fires of human origin.

At lower elevations closed forests which are still intact share many genera with similar forests in Australia and New Guinea and to a lesser degree New Zealand.\footnote{\cite{morat1986floristic}}
Strangling figs may occur as emergents.
In places, particularly on ridge crests, there are also species of \BotanicRef{Araucaria} and \BotanicRef{Agathis}.
A number of endemic palms are prominent as are some tree ferns, including \BotanicRef{Cyathea novaecaledoniae}[Cyathea][novaecaledoniae] with its distinctive greyish trunks which attain heights of up to 30 m.
The ferns \BotanicRef{Asplenium nidus}[Asplenium][nidus] and \BotanicRef{Drynaria rigidula}[Drynaria][rigidula] are frequent as nest epiphytes.

At higher altitudes the montane rain forests are lower in stature, different in species composition and have more New Zealand links.
The strangling figs of lower elevations give way to \BotanicRef{Metrosideros} and the related \BotanicRef{Carpolepis}, some species of which are frequently or occasionally initially epiphytic.
Similarly the nest fern epiphytes give way to \BotanicRef{Astelia novaecaledoniae}[Astelia][novaecaledoniae]. \BotanicRef{Weinmannia} and other members of the Cunoniaceae, tree ferns and conifers of the family Podocarpaceae are important components of these forests.

\BotanicRef{Nothofagus} occurs in patches mixed with other genera or in pure ridge crest stands, although not on all mountains.
The species of \BotanicRef{Nothofagus}, as in New Guinea, belong to the \BotanicRef{Nothofagus brassii}[Nothofagus][brassii] group and some are notable for having unusually large leaves particularly at the juvenile stage\figureref{\fullref{fig:70nothofagus}}.

The most notable physical feature of \IDX{New Caledonia} as far as the plants are concerned is the unusual extent of ultramafic or `serpentine' rocks, which occupy a third of the island.
Unlike some similar sites elsewhere in the world New Caledonian serpentine supports a rich and distinctive flora\footnote{\cite{moratph1986affinites}} in which conifers, \BotanicRef{Nothofagus}, and the families Myrtaceae, Proteaceae, Epacridaceae and Casuarinaceae are prominent.
Shrub and sedge associations are widespread, but there are also quite dense forests particularly at higher elevations.

\subsection[Fiji]{Fiji\thinspace\footnote{\cite{carlquist1965island}}}

Further into the Pacific to the east the Fijian Islands are largely volcanic although, according to some geologists, continental rocks underlie them.
A small continental element in the flora includes ten conifers --- a species of \BotanicRef{Agathis} and several podocarp species in several genera --- and the primitive flowering tree \BotanicRef{Degeneria vitiensis}[Degeneria][vitiensis] placed in its own family close to the Winteraceae.

\subsection[Tahiti]{Tahiti\thinspace\footnote{\cite{carlquist1965island}}}

Much further to the east, entirely volcanic and remote from any continent, Tahiti is the highest and largest island of French Polynesia.
Geologists believe that Tahiti, as well as other similar islands, rose above the sea in isolation.
It would follow that the ancestors of its flora must have arrived by overseas dispersal from already existing islands near and far and directly from continents or indirectly via intervening islands.
Possible involvement of a proposed former continent Pacifica was considered in Chapter 1.

Tahiti, along with other very isolated islands, completely lacks certain plant groups.
It has no conifers, no species of \BotanicRef{Nothofagus}, no members of the Winteraceae or related primitive families nor of the southern family Proteaceae.
Of other important southern families the Myrtaceae is represented by \BotanicRef{Metrosideros} and \BotanicRef{Decaspermum}, the Cunoniaceae by \BotanicRef{Weinmannia} and the Epacridaceae by \BotanicRef{Styphelia}.
Near sea level, native vegetation is virtually non-existent, but at higher levels there is a good cover.
The forest is low, and in hollows and valleys is dominated by \BotanicRef{Weinmannia parviflora}[Weinmannia][parviflora] which, like the New Zealand \IDX{kamahi} (\BotanicRef{Weinmannia racemosa}[Weinmannia][racemosa]), has compound juvenile and simple adult leaves and also often commences life as a low tree fern epiphyte.
A long leaved \BotanicRef{Freycinetia} is abundant and there are small orchid and fern epiphytes.
Among small trees and shrubs are one large and one small-leaved species of \BotanicRef{Myrsine}, a \BotanicRef{Meryta} and in places a large-leaved \BotanicRef{Coprosma} and the small tree \BotanicRef{Fuchsia cyrtandroides}[Fuchsia][cyrtandroides].
In some valleys there are dense tree fern forests which may follow forest clearance as in New Zealand.
The tree ferns are species of \BotanicRef{Cyathea} and at least one has the unusual habit of forming more or less spherical bulbils below the leaf crown which drop off to grow into new plants.

On narrow ridge crests species of \BotanicRef{Metrosideros} are conspicuous along with some of the plants from the valleys and in more open places a small-leaved \BotanicRef{Styphelia}.
One species each of \BotanicRef{Ascarina}, \BotanicRef{Macropiper} and \BotanicRef{Ilex} are also present.
On the ground and sometimes epiphytic is an \BotanicRef{Astelia} and in moist places an \BotanicRef{Elatostema}.

The Tahitian peaks are not high enough for the latitude to support alpine plants.

\subsection[Hawai{\okina}i]{Hawai{\okina}i\thinspace\footnote{\cite{carlquist1970hawaii}}}

The Hawaiian Islands lie in the tropical north Pacific.
The islands in the group are mostly larger than Tahiti and the largest and southernmost, Hawai{\okina}i itself, has dome-like volcanoes, some still active, rising to over 4000 m.
This is high enough to provide alpine conditions.

Hawai{\okina}i shares with New Zealand and Tahiti such genera as \BotanicRef{Metrosideros}, \BotanicRef{Astelia} and \BotanicRef{Coprosma}, but has more species of them than Tahiti.
It lacks \BotanicRef{Weinmannia}, a genus important in some Tahitian and New Zealand forests.
Species and varieties of \BotanicRef{Metrosideros} dominate in most Hawaiian forests, particularly on the island of Hawai{\okina}i where some forms colonise lava flows and may form trees up to \SI{30}{\metre} in height.
In time other less light-demanding and shorter species enter the forests.
They belong to such genera as \BotanicRef{Sapindus}, \BotanicRef{Ilex}, \BotanicRef{Osmanthus} (some unite the Hawaiian species of this genus with \BotanicRef{Nestegis} of New Zealand), \BotanicRef{Pittosporum}, \BotanicRef{Myrsine} and \BotanicRef{Elaeocarpus}.
Tree ferns of the genus \BotanicRef{Cibotium} may be common and sometimes bear a species of \BotanicRef{Cheirodendron} (a genus related to \BotanicRef{Pseudopanax}) as an epiphyte.
There are also a few larger-leaved species of \BotanicRef{Coprosma}, a long leaved \BotanicRef{Freycinetia} and several species of \BotanicRef{Astelia} both terrestrial and epiphytic.
The Lobelia family has undergone remarkable development in Hawai{\okina}i with several genera and many species of shrubs and small trees, some of which are found in moist forests.

Apart from \BotanicRef{Metrosideros} the only other large tree species is \BotanicRef{Acacia koa}[Acacia][koa].
This may be a component of \BotanicRef{Metrosideros} forests or may dominate more open woodlands.

Above tree limit on the high volcanoes of Hawai{\okina}i and Maui there is low shrubby vegetation on well drained sites reminiscent of the mountain shrubland of the volcanic Mt.
Ruapehu of New Zealand.
Close inspection however reveals very few species including a \BotanicRef{Sophora}, a small-leaved \BotanicRef{Styphelia}, a \BotanicRef{Vaccinium} with bright red, shiny berries, a few small-leaved Coprosmas, and a few ferns.

At higher levels still is a volcanic desert with few alpine species --- a few grasses, small ferns, a few herbs in such genera as \BotanicRef{Fragaria} and \BotanicRef{Geranium}, and small shrubs such as the trailing small-leaved \BotanicRef{Coprosma ernodioides}[Coprosma][ernodioides].
The most remarkable alpine is the silversword (\BotanicRef{Argyroxiphium sandwicense}[Argyroxiphium][sandwicense]) of the daisy family which is found only on Mt. Haleakela on Maui.
It is somewhat like the Celmisias in New Zealand with sword-like leaves, but much larger, with the silvery-white leaves almost spherically arranged.
The tall reddish flower heads are not unlike those of some of the Pleurophyllums of the New Zealand subantarctic.

Subalpine and alpine bogs have a much closer cover of plants.
In less wet places there are dwarf shrubs of \BotanicRef{Metrosideros} and other genera, but the wettest parts are essentially cushion bogs with cushions of \BotanicRef{Oreobolus}, some grasses and species of \BotanicRef{Astelia}.
Scattered rosette herbs include \BotanicRef{Drosera} (sundews) and species of \BotanicRef{Plantago}, \BotanicRef{Geranium} and \BotanicRef{Brachycome}.

\section{Temperate South America and Antarctica}

\subsection[South America]{South America\thinspace\footnote{\cite{godley1960botany}}}

Temperate South America, much wider than New Zealand in the north but narrowing southwards, is in some respects like the South Island of New Zealand on a much larger scale.
The Andes and the Southern Alps both lie more or less at right angles to the prevailing westerlies, so climates are wet in the west and much drier in the east.
The Andes are much higher than the Southern Alps in the north at about \ang{34} S, but they diminish in height southwards.

To the east of both mountain ranges are plains, but those in South America are much more extensive and in places drier than those of New Zealand.
To the west in the South Island, and at equivalent latitudes in South America, the mountains come close to the sea with at most a narrow coastal plain in both places.
Further to the north in South America the pattern is somewhat different as here there is a central valley running north-south, which is enclosed by the Andes to the east and a coastal range to the west.
The major difference between the two regions is that in South America there is continuous land from 47 to about \ang{55}S, whereas these latitudes in the New Zealand region are largely occupied by sea with only scattered islands.

The capital of Chile, Santiago, and the northern tip of New Zealand lie at the same latitude, \ang{34}S, but there is little correspondence between the plants of the two places.
The central valley of Chile at this latitude is semi-arid and going northwards the aridity increases until perhaps the driest deserts in the world are reached in Peru.
Going southwards, climates become moister and from about \ang{40}S they are comparable in their range of temperature and moisture regimes to those of New Zealand in the same latitudes.

On the seaward slopes of the coastal ranges from about \SIrange{40}{47}{\degree}S and ranging in altitude from \SIrange{300}{500}{\metre} there is a moist closed forest without \BotanicRef{Nothofagus}.
Similar forests are to be found on the eastern flanks of the coastal range and the western slope of the Andes, although in these cases the deciduous \BotanicRef{Nothofagus obliqua}[Nothofagus][obliqua] is present as an emergent.
This forest type is comparable both vegetationally and floristically to the conifer broadleaf forest of New Zealand, although its conifers in fact are few and inconspicuous.

Compared with forests at similar latitudes in New Zealand there are fewer species overall in the Chilean forests as well as fewer vines and epiphytes.
Shared tree and shrub genera are \BotanicRef{Podocarpus}, \BotanicRef{Weinmannia}, Laurelia, Pseudopanax and Griselinia.
Shared epiphyte genera are Griselinia, Luzuriaga and the ferns \BotanicRef{Asplenium}, \BotanicRef{Grammitis} and \BotanicRef{Hymenophyllum}.
Chilean vines are not related to those of New Zealand.
A sight which is strange to New Zealand eyes is the presence in this South American forest type of clumps of the bamboo \BotanicRef{Chusquea quila}[Chusquea][quila] and the substitution for our asteliad nest epiphytes of the bromeliad \BotanicRef{Fascicularia bicolor}[Fascicularia][bicolor] and for our ground astelias another bromeliad \BotanicRef{Greigia sphacelata}[Greigia][sphacelata].
These bromeliads can be considered southern outliers of a family of specialised epiphytes that is richly developed in the Amazon rain forests.

\BotanicRef{Nothofagus} forests are also extensively developed in southern South America.\footnote{\cite{mcqueen1976ecology}}
There are 10 species of \BotanicRef{Nothofagus} altogether, 2 belonging to the \IDX{silver beech}[beech!silver] (\BotanicRef{Nothofagus menziesii}[Nothofagus][menziesii]) group and 8 to the \IDX{red beech}[beech!red] (\BotanicRef{Nothofagus fusca}[Nothofagus][fusca]) group.
Seven species are deciduous (both of the \BotanicRef{Nothofagus menziesii}[Nothofagus][menziesii] group species and 5 of the \BotanicRef{Nothofagus fusca}[Nothofagus][fusca] group), generally in response to seasonal drought or cold.

Above the lowland forest on the seaward slopes of the coastal range there is first a zone of evergreen \BotanicRef{Nothofagus} involving two species in succession, then below treeline the deciduous \BotanicRef{Nothofagus antarctica}[Nothofagus][antarctica].
This is similar to the \BotanicRef{Nothofagus cunninghamii}[Nothofagus][cunninghamii]/\BotanicRef{Nothofagus gunnii}[Nothofagus][gunnii] pattern in Tasmania.
On the inner coast range and the western Andes the pattern is different.
Above the lowland forest the zonation sequence is deciduous, evergreen, then deciduous \BotanicRef{Nothofagus}.

The two deciduous species of the \IDX{silver beech}[beech!silver] (\BotanicRef{Nothofagus menziesii}[Nothofagus][menziesii]) group, unlike \BotanicRef{Nothofagus menziesii}[Nothofagus][menziesii] itself, are tolerant of some seasonal drought and used to form forests in the central valley from \SIrange{38}{41}{\degree}S and further north at higher altitudes on the coastal range. \BotanicRef{Sophora microphylla}[Sophora][microphylla] and species of \BotanicRef{Aristotelia} and \BotanicRef{Muehlenbeckia} are sometimes associated with these species of \BotanicRef{Nothofagus}.

Conifers associated with \BotanicRef{Nothofagus} forests are \BotanicRef{Podocarpus} species in the subcanopy, \BotanicRef{Saxegothea conspicua}[Saxegothea][conspicua] in the canopy and \BotanicRef{Araucaria araucana}[Araucaria][araucana] as an emergent near treeline in \BotanicRef{Nothofagus pumilio}[Nothofagus][pumilio] forests.

Going southwards in South America beyond New Zealand latitudes \BotanicRef{Nothofagus} forest continues as the only type of forest to \ang{55}S or about the latitude of Macquarie Island.
Only three species of \BotanicRef{Nothofagus} are present in these latitudes --- the evergreen \BotanicRef{Nothofagus betuloides}[Nothofagus][betuloides] and the deciduous \BotanicRef{Nothofagus pumilio}[Nothofagus][pumilio] and \BotanicRef{Nothofagus antarctica}[Nothofagus][antarctica].

From about \SIrange{52}{55}{\degree}S the sequence from west to east is: an open moorland with little forest; evergreen \BotanicRef{Nothofagus betuloides}[Nothofagus][betuloides]; deciduous \BotanicRef{Nothofagus pumilio}[Nothofagus][pumilio] and \BotanicRef{Nothofagus antarctica}[Nothofagus][antarctica]; and short tussock grassland on the eastern plains.
It has been suggested that the lack of forest in the west is due to the cloudy and thus cooler climate.\footnote{\cite{mcqueen1976ecology}}
As conditions become drier and somewhat warmer to the east, evergreen \BotanicRef{Nothofagus} forest can be supported, then deciduous \BotanicRef{Nothofagus} forest and finally in the driest conditions of all only grass and shrubland.
Alternatively, it has also been proposed that the lack of forest in the east is due to the diorite rocks there which form a thin infertile soil.\footnote{\cite{godley1960botany}}

The moorlands are of particular interest as they have cushion bogs very similar ecologically to those of New Zealand and Tasmania and with related floras.
Genera shared with New Zealand are \BotanicRef{Oreobolus}, \BotanicRef{Gaimardia}, \BotanicRef{Phyllachne}, \BotanicRef{Donatia}, \BotanicRef{Astelia}, \BotanicRef{Drosera} and \BotanicRef{Lepidothamnus}. \BotanicRef{Lepidothamnus fonckii}[Lepidothamnus][fonckii] has a very similar growth habit to the \IDX{pygmy pine} (\BotanicRef{Lepidothamnus laxifolius}[Lepidothamnus][laxifolius]) of New Zealand.

The short tussock grasslands of the eastern plains are similar in general appearance to those of the eastern South Island and also involve species of \BotanicRef{Festuca}.
The associated shrubs and herbs are mostly unrelated to those of New Zealand with the exception of small species of \BotanicRef{Discaria} (\IDX{matagouri} in New Zealand).

Above treeline in southern South America are extensive alpine areas.\footnote{\cite{moore1975alpine}}
There does not seem to be a well developed mountain shrubland zone as in some New Zealand mountains, although reduced, shrubby forms of \BotanicRef{Nothofagus antarctica}[Nothofagus][antarctica] may play that role.
In poorly drained places cushion bogs similar to those already described occur, and on rocky sites fellfield with a predominance of cushion plants including large specimens of \BotanicRef{Azorella} and the related \BotanicRef{Bolax}.
In the north where the Andes are at their highest there are extensive screes with some specialised plants similar to those of New Zealand, but not related to them.\footnote{\cite{goodspeed1950plant}}

\subsection[Antarctica]{Antarctica\thinspace\footnote{\cite{wace1965vascular}}}

The southern tip of South America and the tip of the Antarctic Peninsula (Grahamland) of Antarctica are only about \SI{800}{\kilo\metre} apart and are connected by a loop-like submarine ridge bearing several islands.
Unlike adjacent South America, however, the present day plants of Antarctica can be dismissed quite briefly.
Moss cushions and lichens grow on exposed rock on the Antarctic Peninsula and some rocks free of ice on the main part of Antarctica, particularly in the strange `Dry Valleys' near the Ross Sea.
The only flowering plants are two small species on the Antarctic Peninsula: a grass, \BotanicRef{Deschampsia antarctica}[Deschampsia][antarctica], and a chickweed relative, \BotanicRef{Colobanthus quitensis}[Colobanthus][quitensis], both of which are also found at the southern end of South America and elsewhere.
