\chapter{Where Did Our Alpine Plants Come From?}

Despite the fact that the alpine flora in New Zealand is considerably richer in species than the forest flora, its derivation poses the more puzzling problem.
This is because geological evidence indicates that the present high mountains with their cold climates are relatively recent.
In fact, for all except the last few of the past 100 million years it is suggested that New Zealand had warmer climates than now and much lower relief.
In these circumstances it is difficult to imagine any habitats suitable for cold-tolerant plants that could have formed a nucleus for the present diverse alpine flora.

\section{Derivation from Plants Already Here}

Cockayne suggested the possibility that a few alpine species from the much earlier mountain building period of about 100 million years ago (Rangitata Orogeny) may have survived through the warmer, mountainless times in suitable sites such as rocky bluffs.\footnote{\cite{cockayne1928vegetation}}
He described these species as `plastic', that is, that although requiring open habitats, they would have had a wide temperature tolerance.
With the coming of the recent Ice Age and high mountains these surviving `alpines' or their derivatives would have diversified into the new cold habitats.

Present rocky coast species of otherwise alpine genera could appear to support Cockayne's hypothesis.
Some of these are: the \IDX{spaniard} (\BotanicRef{Aciphylla squarrosa}[Aciphylla][squarrosa]), of the Cook Strait coasts; \IDX{mountian aniseed} (\BotanicRef{Gingidia montana}[Gingidia][montana]) on the Marlborough coast; \IDX{Akaroa daisy} (\BotanicRef{Celmisia mackaui}[Celmisia][mackaui]) of Akaroa, Banks Peninsula and \IDX{Lindsay's daisy} (\BotanicRef{Celmisia lindsayi}[Celmisia][lindsayi]) of the south-east Otago coast; \IDX{Lyalls carrot} (\BotanicRef{Anisotome lyallii}[Anisotome][lyallii]) of the south-east Otago coast, Fiordland and Stewart Island; and \IDX{wharariki} (\BotanicRef{Phormium cookianum}[Phormium][cookianum]), perhaps equally common on coastal rocks as in the high mountains.
The problem here is that these are all herbaceous flowering plants and 100 million years ago fossil evidence indicates that flowering plants, then in the early stages of their evolution, were predominantly woody.
If the fossil record can be taken as a reliable guide (and some question this) it was not until long after the time of Rangitata Orogeny in higher northern (and perhaps also southern) latitudes that herbaceous flowering plants began to expand and diversify.

However, a modification of Cockayne's theory might be tenable.
After herbaceous angiosperms became common some of them may have reached New Zealand by long distance dispersal from the northern hemisphere, and perhaps from a then warmer Antarctica, and, if tolerant of warmer temperatures, they may have established in various open lowland habitats.
With the coming of alpine habitats these species would have been able to move into them and diversify.

This is not to say that the lowland species of herbaceous, predominately alpine genera in New Zealand are necessarily primitive and ancestral.
During glacials, alpine vegetation probably extended at times right to the coasts, particularly in southern New Zealand.
With each interglacial, forests would replace alpine plants in the lowlands, except on some exposed rocky coastal sites.
Here some alpine species could persist and even give rise to new forms.
In these circumstances we would have to regard the coastal species as derived from alpine species rather than as being their ancestral stock.

Wardle\footnote{\cite{wardle1963evolution}} suggested that in warmer pre-Ice Age times New Zealand may have extended much further south to the limits of what is now the submerged Campbell Plateau.
Even in those warmer times he felt that such a southern extension would have had a cool temperate climate able to support a flora `which would later give rise to the present mountain flora'.
He based this idea on the present, fairly diverse, mostly herbaceous floras of the subantarctic islands, which share a number of genera with the New Zealand mountains, but which also have a number of endemic genera which, Wardle suggests, did not reach the main islands.

Unfortunately geologists do not support the idea of such a southern extension at that time, although even scattered islands to the south in warmer times may have provided adequate sites for the evolution of a stock of cool temperate plants.

In a later paper Wardle\footnote{\cite{wardle1968evidence}} proposed that a small element of the alpine flora may have been present in New Zealand throughout the warmer climates of the Tertiary, a view partly supported by fossil evidence.
The generic and subgeneric groups concerned:

\begin{quote}
	are small and taxonomically isolated, restricted to cool habitats, and either endemic to New Zealand or occurring elsewhere only in the Australian or American sectors of the South Temperate Zone.
\end{quote}

Wardle thinks it unlikely that the species of this element have been derived from warm-climate Tertiary ancestors, or that they reached New Zealand by long distance dispersal.
He lists 29 flowering species in 15 genera which are mostly confined to cool, wet, infertile soils.
Of these, \BotanicRef{Nothofagus} (\BotanicRef{Nothofagus menziesii}[Nothofagus][menziesii] --- \IDX{silver beech}[beech!silver]) is a forest dominant, \BotanicRef{Stilbocarpa} is a tall herb of far southern coasts, while the remainder are low growing plants ascending to subalpine or alpine levels, eight of them being more or less cushion or mat-forming'.
Genera with species of the latter form are \BotanicRef{Hectorella}, \BotanicRef{Phyllachne}, \BotanicRef{Donatio}, \BotanicRef{Tetrachondra} and \BotanicRef{Oreobolus}.
He also suggests that certain shrubby, cold-tolerant, higher altitude conifers such as \IDX{snow totara} (\BotanicRef{Podocarpus nivalis}[Podocarpus][nivalis]), \IDX{pygmy pine} (\BotanicRef{Lepidothamnus laxifolius}[Lepidothamnus][laxifolius]) and \BotanicRef{Phyllocladus aspleniifolius var.\ alpinus}[Phyllocladus][aspleniifolius var.\ alpinus] may also have been present through the Tertiary rather than having been derived more recently from taller, lower altitude relatives.
This view is based on a study which indicates that the shrubby forms are more primitive cytologically than the trees, and, if the cytological interpretation is valid, it follows that the former are unlikely to have been derived from the latter.

It had been earlier suggested\footnote{\cite{fleming1963age}} that this alpine element may have reached the New Zealand mountains from Antarctica, as presumably would also have been the case for the comparable elements in Tasmania and South America.
For New Zealand this would imply a migration via the subantarctic islands, but on the whole Wardle does not favour this view as only 6 of the 18 genera concerned occur on these islands, and in the case of \BotanicRef{Nothofagus} poor dispersal ability would suggest that crossing such wide ocean gaps would not have been possible.
So if species of this element were already present in New Zealand before colder climates and mountains developed, where did they grow? Wardle points out that even today a number of the species of the group occur in the wet Westland climate within \SI{300}{\metre} of sea level on infertile soils, and in Hawai{\okina}i, whose climate may be similar to that of New Zealand during the Tertiary, otherwise alpine species can be found in a bog at only \SI{600}{\metre} altitude.

\begin{quote}
	If, during the period of Tertiary warmth (in New Zealand), there were infertile soils on peneplained uplands drenched by persistent mist and rain, some cool climate plants would undoubtedly exist … On the other hand, considerably colder conditions would seem necessary for the Tertiary existence of the high altitude genera hectorella, \BotanicRef{Haastia}, \BotanicRef{Phyllachne}, \BotanicRef{Donatia} and \BotanicRef{Rostkovia}, and it is perhaps in respect to these that the suggestion of immigration from Antarctica at the end of the Tertiary is most attractive.
\end{quote}

Finally, Wardle suggests, the steep slopes with young soils of most of the present mountains would not have been suitable for this element, so it has remained as a minor component of the alpine flora as a whole.

The possibility of the derivation of alpine species from warm-climate forest ancestors in New Zealand has already been mentioned.
Most authors would accept this as one source, but most also consider its contribution as relatively minor --- involving mostly subalpine and alpine shrubs.
For example, some botanists consider that divaricate shrubs are basically adaptations to Ice Age cold and drought and some of these, which belong to such genera as \BotanicRef{Coprosma}, \BotanicRef{Pittosporum}, \BotanicRef{Melicytus}, \BotanicRef{Myrsine}, \BotanicRef{Aristotelia}, are to be found in subalpine and alpine habitats.
In \BotanicRef{Coprosma} and \BotanicRef{Myrsine} there are even a few completely prostrate alpine species.
Among non-divaricate derivatives, \BotanicRef{Dracophyllum} has a number of needle-leaved and some broader-leaved shrubs in lower alpine and near cushion forms in higher alpine situations and \BotanicRef{Hebe} is well represented by alpine shrubs, some of which are of whipcord form.
\BotanicRef{Astelia} is an herbaceous genus with a number of forest floor species and one epiphytic species, which is also quite well represented in the alpine zone.

\section{Immigrants from The Northern Hemisphere}

However, if it is true that only a minor part of the New Zealand alpine flora is derived from plants that were already present here, where did the majority of the alpine species come from? Raven\footnote{\cite{raven1973evolution}} suggests that many of them are derived from ancestors which reached New Zealand by long distance dispersal after the mountains had formed here.
He believes that the ultimate source of the genera to which these ancestral colonisers belonged was temperate Asia and that they were only able to reach Australasia in recent geological times after mountains, contemporary with those of New Zealand, had formed in south east Asia, Indonesia, New Guinea and eastern Australia.
Species of these genera `mountain hopped' across the tropics to eastern Australia and then across to New Zealand by the agency of the strong, frequent, westerly winds of the `Roaring-Forties'.
Raven explains the fact that some of the genera concerned have many more species in New Zealand than in eastern Australia by the higher, ecologically more diverse mountains in New Zealand providing a setting for extensive radiation of the immigrants after their arrival from Australia.
In view of the direction and strength of the prevailing winds he regards migration from New Zealand to Australia as improbable.

Some of the genera that Raven believes reached New Zealand in this way are represented here by only one or a few species, for example, \BotanicRef{Anemone}, \BotanicRef{Geranium}, \BotanicRef{Linum}, \BotanicRef{Stellaria} and \BotanicRef{Viola}, but others have dozens of species and include \BotanicRef{Epilobium}, \BotanicRef{Ranunculus}, \BotanicRef{Myosotis}, the grass genus \BotanicRef{Poa}, the sedge \BotanicRef{Carex} and the rush genus \BotanicRef{Juncus}.
Other large genera are not known in north temperate regions, but at least some of their species were originally placed in northern genera from which they may have evolved within Australasia, for example, \BotanicRef{Celmisia} (Aster), \BotanicRef{Anisotome} and \BotanicRef{Aciphylla} (Ligusticum), and \BotanicRef{Hebe} (Veronica).

In a response to Raven's theory Wardle\footnote{\cite{wardle1978origin}} claims that Raven underestimates the possibility of plants dispersing from New Zealand to Australia.
He points out that vigorous easterly winds can flow from New Zealand to Australia when a tropical depression meets an anticyclone over New Zealand and also notes that there are several bird species migrating annually between New Zealand and Australia, which might serve as seed transporters.
There is also the possibility of direct transport between New Zealand and northern Asia and north-west America by a number of migratory birds.
A few mountain species of grasses, which are the same as or closely related to north temperate species may have arrived in this way.

Wardle further states that it is necessary to know something about patterns of evolution within a genus before one can hypothesise about directions of migration.
If for instance there were in a genus a mixture of primitive and advanced species in New Zealand, but only a few advanced species in Australia, then it is likely that the latter would have been derived from New Zealand.

The patterns in \BotanicRef{Epilobium} indicate that Raven's interpretation could be correct with the exception of two species in Australia which may have been derived from New Zealand.
However, since Wardle's article, abundant fossil pollen of \BotanicRef{Epilobium} has been found in New Zealand of Oligocene age, a period long before the formation of the present mountains.
So if \BotanicRef{Epilobium} persisted in New Zealand until colder climates developed, then some at least of the present species may derive from that early stock, though others may have come from more recent colonisations.

In \BotanicRef{Ranunculus} the affinities of the lowland and montane species are with Australia, but the alpines are most like species in South America suggesting a different origin.

Wardle suggests that an ultimate South American derivation is also possible for the snow grass genus \BotanicRef{Chionochloa}, which exhibits cytological and other similarities with \BotanicRef{Cortaderia} (which includes Pampas grass) of South America and New Zealand.
Similarly most of the New Zealand species of \BotanicRef{Plantago} belong to an otherwise South American section of the genus and so may have been derived from there rather than from temperate Asia via Australia.

\section{Southern Origins of Some Alpine Genera}

I now want to consider some of the southern genera thought by some botanists to be related to and probably derived from north temperate genera.
I propose an alternative hypothesis --- that they may have evolved independently in the southern hemisphere.

With regard to the derivation of herbaceous genera in general it is worthwhile to consider what the vegetation pattern of the world may have been before this growth form suited to cooler, seasonal climates became widespread.
The fossil evidence suggests that at these earlier times rain forest extended from the equator half way to the poles or even further and we can speculate that within such forests there would have been the woody ancestors of the herbaceous genera.
Most of the latter would have originated in the northern hemisphere because of the greater area of land there and the more seasonal climates, but there seems no reason why other herbaceous genera could not have been derived in the southern hemisphere also from woody stock at the fringes of the great rain forest belt.

\subsection{New Zealand Umbelliferae}

In an earlier review\footnote{\cite{dawson1971relationships}} I proposed two evolutionary lines within the more notable New Zealand genera of the main subfamily of the Umbelliferae (carrot family).
One included \BotanicRef{Aciphylla} (spaniards) and \BotanicRef{Anisotome} and is distinguished by the inflorescences arising at the centre of leaf rosettes and terminating their growth and by their dioecism (separate male and female plants).
The other group, comprising \BotanicRef{Scandia}, \BotanicRef{Gingidia} and \BotanicRef{Lignocarpa}, has mostly lateral inflorescences and is mostly gynodioecious (separate female and hermaphrodite plants).
In the latter group I considered \BotanicRef{Scandia} to be the most primitive genus as it is semi-woody with extended stems.
\IDX{Koheriki}[koheriki] (\BotanicRef{Scandia rosaefolia}[Scandia][rosaefolia]) is the more shrubby of the two species of the genus and it is restricted to the northern half of the North Island and there often to mild coastal habitats.
It is not difficult to imagine this species, and perhaps even woodier, now extinct, relatives, existing in New Zealand in warm pre-Ice Age times.
From \BotanicRef{Scandia} the completely herbaceous montane to alpine \BotanicRef{Gingidia} could have evolved and also the specialised \BotanicRef{Lignocarpa} of shingle slips.
Since this idea was proposed a probable hybrid between \IDX{spaniard} (\BotanicRef{Aciphylla squarrosa}[Aciphylla][squarrosa]) and \IDX{mountain aniseed} (\BotanicRef{Gingidia montana}[Gingidia][montana]) has been discovered in Marlborough\footnote{\cite{webb1984natural}} which suggests that the two lines are more closely related than I thought and that all the genera concerned may have been derived from woody ancestors with a long history in New Zealand.
In eastern Australia there are two species assignable to \BotanicRef{Aciphylla}, one to \BotanicRef{Anisotome} and three to \BotanicRef{Gingidia}, including a localised occurrence of \BotanicRef{Gingidia montana}[Gingidia][montana].
In view of the foregoing discussion it seems more likely to me that the Australian species have a New Zealand ancestry than vice versa.

In a recent review Webb\footnote{\cite{webb1986breeding}} gives a largely opposite interpretation.
He considers the few Australian species of these genera to be primitive, while New Zealand has a range from primitive to specialised forms.
In contrast to my view, he regards the woodiness of \BotanicRef{Scandia} as a recent specialisation.
With one exception, he suggests that primitive species of these genera migrated from Australia to New Zealand with consequent diversification and specialisation here.
The exception is \IDX{mountain aniseed} (\BotanicRef{Gingidia montana}[Gingidia][montana]), where the localised occurrence in northern New South Wales he considers to be a case of recent migration from New Zealand to Australia.

Of course, at an earlier time before high mountains existed in New Zealand, there could have been only primitive species of these genera in both Australia and New Zealand and migration could have been in either direction.
With the formation of the high mountains in New Zealand more specialised forms would then have evolved here.

According to yet another view some would argue that long distance migrations have not been involved at all.
The occurrence of the same or related alpine plants in both Australia and New Zealand would derive from the time when they shared the same region before it was sundered by the drifting away of the New Zealand crustal complex from Australia.

With little direct evidence from the past we can only speculate.

\subsection{Celmisia}

As well as the 60 or so species of \BotanicRef{Celmisia} in New Zealand there are two related genera in the subantarctic islands: the semi-woody \BotanicRef{Damnamenia} and the large, tufted, herbaceous \BotanicRef{Pleurophyllum}.\footnote{\cite{given1973damnamenia}}
\BotanicRef{Pleurophyllum} is considered to be intermediate in some respects between \BotanicRef{Celmisia} and one group of the woody genus \BotanicRef{Olearia}, also strongly represented in New Zealand.
This fact, plus possible wild hybrids between \BotanicRef{Olearia} and \BotanicRef{Celmisia}, suggests a derivation, perhaps within New Zealand, of \BotanicRef{Celmisia} from \BotanicRef{Olearia}.
Some of the New Zealand celmisias are described as subshrubs, but the five species in east Australia are herbaceous and in Wardle's words `seem to represent end points of evolutionary pathways stemming from New Zealand'.

\subsection{Hebe and Related Genera}

New Zealand has perhaps 100 species of \BotanicRef{Hebe} and there are only three species elsewhere.
Two of these in southern South America are shared with New Zealand (\IDX{kokomuka}, \BotanicRef{Hebe elliptica}[Hebe][elliptica], and \IDX{koromiko}, \BotanicRef{Hebe salicifolia}[Hebe][salicifolia]) and the third is \BotanicRef{Hebe rapensis}[Hebe][rapensis] on Rapa Island in French Polynesia.
The related but generally less woody \BotanicRef{Parahebe} is also represented in eastern Australia and New Guinea and \BotanicRef{Chionohebe}, probably derived from \BotanicRef{Parahebe}, is a genus of small, high alpine cushion plants of which two of the New Zealand species extend to south-eastern Australia.
\BotanicRef{Veronica}, from which Raven\footnote{\cite{raven1973evolution}} would consider the \BotanicRef{Hebe} alliance to have been derived, is a mostly north temperate, herbaceous to slightly woody genus.
It seems reduced and specialised and so is unlikely to have been ancestral to the woody hebes, some of which may become small trees with trunks sometimes \SI{30}{\centi\metre} in diameter.
There are differences too between the basic chromosome numbers of \BotanicRef{Veronica} and the \BotanicRef{Hebe} alliance which do not suggest a close relationship.
There are indeed a few true veronicas in eastern Australia but, as Wardle\footnote{\cite{wardle1978origin}} comments, `it seems … possible that the handful of Australian true veronicas derive from recent northern immigrants that are not directly related to the \BotanicRef{Hebe} alliance'.

Presumably \BotanicRef{Veronica} was originally derived from woody, northern hemisphere ancestors, which may have been related to and perhaps also ancestral to the \BotanicRef{Hebe} alliance.

Once again it is clear that much more research is required into a number of genera before we can attain any degree of certainty about the relative proportions of New Zealand alpines that have been derived from:

\begin{description}
	\item[{(a)}]geologically recent immigrants from the northern hemisphere;
	\item[{(b)}]more ancient immigrants from the northern hemisphere; or
	\item[{(c)}]genera of southern hemisphere origins.
\end{description}
