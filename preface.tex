\markboth{}{}

\chapter*{Foreword}

{\itshape%
    This resource is unavailable for copyright reasons, and cannot be made available until such time as permission to reproduce this work online is received from the rights owner.

    If you have information about the rights owner, or about their willingness to make this resource available online, please contact the NZETC Director.
}

\cleardoublepage%
{
    \thispagestyle{empty}
    \hspace{0pt}
    \vfill
    \begin{center}
        Dedicated to\par
        Hugh Douglas Gordon\par
        Professor of Botany, Victoria University\par
        1947–1977
    \end{center}
    \vfill
    \vfill
}

\chapter*{Preface}

It is now 60 years since the third edition of Cockayne's New Zealand Plants and their Story was published.
A fourth edition, edited by E. J. Godley, appeared in 1967 with the text virtually unchanged, but with updated botanical names and references in footnotes to more recent information.
In the preface to this edition Godley says:

\begin{quote}
    Dr Leonard Cockayne died in Wellington on 8 July 1934 at the age of 79, seven years after the third edition of this book appeared.
    The publication of a fourth edition would probably have pleased him, in showing that his story was not forgotten and in making it again widely available.
    But I think too that he would have criticised us --- for he was a forthright man --- for not having produced something to take its place.
\end{quote}

The aim of the present book is not to replace Cockayne's account, which, as Godley states, is `a classic of New Zealand botanical literature', but to build on it by reviewing the increased knowledge of New Zealand plants and their communities, living and fossil, gained over the past 60 years.
As with \emph{New Zealand Plants and their Story}, identification of species is not the primary purpose of this book.
A number of guides to the identification of various groups --- ferns, trees and shrubs, alpines --- already meet that need.
The emphasis is on a representative selection of species, their life styles, their interesting and peculiar features, their histories and relationships, presented within the framework of their communities.
In addition, general characteristics and peculiarities of the flora and theories relating to them are reviewed.

As is the case with most general accounts of the floras of countries, this book is chiefly concerned with the larger and more conspicuous vascular plants --- ferns and their relatives, conifers and flowering plants.
The generally smaller mosses, liverworts, lichens, fungi and seaweeds are equally numerous and diverse in New Zealand, but that is a story that others are better qualified to tell.

I hope that this book will be useful to the interested layperson, to students and to botanists both in New Zealand and overseas.
I have tried to avoid technical terms as much as possible, but the use of botanical names for species, as well as common names where these are well-established, seemed unavoidable.
The overseas botanist is unlikely to find `supplejack' or `bush lawyer' very meaningful, but \BotanicRef{Ripogonum scandens}[Ripogonum][scandens] and \BotanicRef{Rubus cissoides}[Rubus][cissoides] will convey a great deal about our species from features shared with their relatives elsewhere.

The book derives in large part from courses given at both graduate and undergraduate levels and also over a number of years to Workers' Educational Association classes.

\section*{The Book And Author}

\emph{Forest Vines to Snow Tussocks} is the first general review of the distinctive features of New Zealand native plants since Cockayne's \emph{New Zealand Plants and their Story}, now long out of print.

The plant communities, from dense lowland forest to alpine tundra, provide the book's framework, and within this many interesting and often puzzling aspects are considered: vines, epiphytes and parasites of the forest; the many peculiar shrubs with minute leaves and densely interlaced twigs; and the dense cushion plants, or `vegetable sheep', of the high mountains.

Although more than 80 percent of New Zealand's native plants are found nowhere else our flora does not stand alone, so comparisons are made with the plants of other southern lands.
In the last chapter an attempt is made to reconstruct the history of the New Zealand flora over geological time.

\emph{Forest Vines to Snow Tussocks} fills an important gap in New Zealand's botanical literature — a fascinating and invaluable book for student and general reader alike.

John Dawson was born in 1928 in Eketahuna, where his interest in plants began.
He was educated at Victoria University and the University of California at Berkeley, and since 1957 has taught in Victoria University's Botany department.
He has a long standing interest in the characteristics, history and relationships of New Zealand plants, and has conducted detailed research into a family of alpine plants (Umbelliferae).
For the past 25 years he has also been involved in studies of the New Caledonian flora, with detailed research, soon to be published in two volumes, of the family Myrtaceae.

\chapter*{Acknowledgements}

I am very grateful to a number of botanists who have made useful suggestions on the whole manuscript or parts of it.
E. J. Godley, B. V. Sneddon, W. R. Philipson and B. P. J. Molloy reviewed the entire manuscript; F. B. Sampson, D. R. McQueen and G. M. Rogers read some early chapters; P. S. Green, P. Williams, W. R. Sykes and J. Campbell advised on Chapter 10 and D. Pocknall, D. Mildenhall and A. M. Holden on Chapter 12.

For technical assistance I am especially grateful to Maureen Cooper for word processing the several revisions, to John Casey, his predecessor M. D. King and the Photographic Facility for taking some of the photographs and for printing most of the others, to Philippa Spackman for preparing several maps and diagrams, and to the editorial staff of Victoria University Press.
